\documentclass[UTF8]{ctexart}
\usepackage{verbatim}
\begin{document}
\pagestyle{headings}
\title{自我总结}
\author{孔畅
%\thanks{Professer Zhao}
}
\maketitle
\begin{abstract}
2019暑假集训及2019春夏集训总结
\end{abstract}

\section{强项与弱项}

强项主要在于数学方向,毕竟数学系,也就数学水平还能跟上大家。除了数学相关(代数、几何、概率、组合、博弈)能有平均水平之外,其他算法都是弱项,高中没有学过算法,现在还有很多知识需要自学。

\section{集训总结}

\subsection{2019春夏集训总结}

训练还算认真,但水平实在有限,经常集训队倒数,最终发挥不错从校赛挤进了省赛。

个人作用主要是先看题,写点简单的签到,然后开始想数学相关的题推导公式,或者帮队友推公式。比赛后半段一般自己开一些构造题或者几何题,校赛和省赛运气都不错,都在后期把构造题做出来了,帮队伍拿到了关键分。

\subsection{2019暑假集训总结}

暑假到现在为止都是个人赛,算是非常明显地体现出了我的不足,常规算法基础不行,经常有大家都能过的题不会做,仅有的打得比较好的一场是因为数学优势,个人实力还有待加强。

\section{出题组相关}

我属于出题组B。由于会的东西还是太少,只搞了半年算法也没有做过很多题,所以没有参与出题。只帮忙想了一下思路,后来验了几个简单题。同组成员都非常负责,题目质量也都很高,非常佩服他们。

\section{组队情况}

\subsection{春夏集训队友评价}

春夏集训中我的队友是黄彦玮和夏天鑫。

黄彦玮实力是三个人中最强的,作为主代码手,学习热情也很高,经常向我们分享学的新算法。夏天鑫算法基础扎实,也是队伍重要的一员,一般在黄彦玮卡题时接手键盘。

\subsection{组队意向}

没有特别的组队意向,感觉我还是要以个人实力提升为主,老队友相处很不错,新队友也不排斥。

\end{document}